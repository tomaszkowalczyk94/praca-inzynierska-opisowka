\chapter{Testy}
		
	\section{????}
	Bardzo ważną kwestią w projekcie było dokładne pokrycie kodu aplikacji w testach jednostkowych. Emulator mikro-kontrolera to specyficzna aplikacja. Z pozoru mało znaczący błąd może sprawić że emulator stanie się bezużyteczny. 
	
	Dla przykładu, jeśli dla 3 bajtowego rozkazu procesora zwiększymy rejestr PC o 2 zamiast o 3, to nie wykona się następna instrukcja przewidziana przez programistę. Dalsza praca emulatora stanie się nieprzewidywalna, a następna instrukcja całkowicie “wykolei” nasz program który zacznie wykonywać losowe instrukcje. 
	
	Dlatego poprawne wykonanie każdego rozkazu jest tak ważne w moim projekcie. Aby uchronić się przed tego typu prostymi błędami każdy emulowany rozkaz posiada swój test/testy jednostkowe napisane przy pomocy biblioteki Junit. 
	
	Przykładowy test dla rozkazu LD A, I. Rozkaz ten ładuje zawartość rejestru A z I:
	\lstinputlisting[language=Java]{listings/exampleTest.java}
	
	Opisany przykład ukazuje że prosta z pozoru operacja jak pobranie wartości jednego rejestru i przeniesienie go do innego, wymaga objętościowych testów. Oprócz testowania czy poprawna wartość znajduje się w rejestrze docelowym, musimy sprawdzić także czy flagi CPU zostały ustawione na poprawnych wartościach, czy ilość przewidywanych cykli zegara została poprawnie zwiększona, czy rejestr PC został zainkrementowany. 
	
	\section{Test-driven development}
	TDD to metoda pisania oprogramowania. Zakłada ona że test jednostkowy dla danej funkcjonalności powstaje jako pierwszy. Dopiero po napisaniu testu implementujemy kod programu, a następnie testujemy za pomocą już napisanych testów. Za pomocą TDD była pisana cała aplikacja
\chapter{Implementacja}
	%fragmenty kodu, podział na pakiety
	
	W tym rozdziale przedstawiono opis implementacji projektu,  



	\section{XBit}
	XBit to projekt mający za zadanie ułatwić operacje bitowe w javie, która interpretuje liczby całkowite w kodzie dopełnień do dwóch, i nie jest możliwa interpretacja ich w naturalnym kodzie binarnym (w skrócie NKB). Z tego powodu powstał projekt XBit implementujący taką opcję. Potrafi on odczytać liczbę n-bitową, i zwrócić jej wartość jako zmienna integer, interpretując ją jako bity w U2 lub NKB.
	
	
	
	Wewnątrz obiektów reprezentujących liczby binarne, wartość przechowywana jest w zmiennej typu integer, którą nazwano "valueContainer". Może ona więc przyjmować teoretycznie wartości od -2147483648 do 2147483647. W praktyce liczba ta nigdy nie jest ujemna, może przyjąć wartości od 0 do $ 2^{n}-1 $ gdzie n oznacza liczbę bitów. 
	
	Dla przykładu,  valueContainer obiektu Xbit8, który w zamierzeniu ma przechowywać wartości 8-bitowe, będzie mógł zawierać wartość od 0 do $ 2^{8}-1 = 255 $. 
	 	 
	Jeśli XBit8 będzie przechowywał liczbę binarną 11111111 (czyli odczytując w NKB to decymalne 255, a w u2 decymalnie -1). Jego wartość zmiennej valueContainer będzie wynosić binarnie 
	00000000 00000000 00000000 11111111. (czyli odczytując zarówno w NKB jak i w U2 będzie to 255). Niestety nadmiarowość bitów jest w tym rozwiązaniu wymagana, ponieważ wartość w kodzie u2 której najbardziej znaczącym bitem wynosi 1, jest ujemna. 
	
	Podsumowując, cała sztuczka XBit wykorzystuje fakt takiego samego zapisu liczb dodatnich w U2 i NKB, jeśli tylko będą one zapisane na większej ilości bitów niż wymagana.
	 
	
	\subsection{Implementacja klasy Xbit}
	XBit to klasa abstrakcyjna zawierająca wspólne elementy dla klas XBit8 i XBit16 oraz, ewentualnych przyszłych klas. Poniżej opiszę warte uwagi metody.
	
	\subsubsection{boolean getBit(int index)}
	
	\begin{listing}[h]
		\begin{minted}{java}
public boolean getBit(int index) {
	if(index <0 || index > getSize()-1) {
		throw new NumberFormatException();
	}
	return ((valueContainer >> index) & 1) == 1;
}
		\end{minted}
		\caption{Metoda boolean getBit(int index)}
		\label{listing:xbitMethodGetBit}
	\end{listing}
	W kodzie \ref{listing:xbitMethodGetBit} zaprezentowano metodę getBit zwracającą bit o danym indeksie przekazanym w parametrze. W linii nr 2 sprawdzana jest poprawność podanego indeksu. Jeśli parametr jest niepoprawny, w linii nr 3 wyrzucany jest wyjątek NumberFormatException. W linii nr 5 wykonujemy przesunięcie bitowe pola valueContainer o podany index. Ponieważ wynikiem przesunięcia bitowego jest liczba o typie integer, wykonujemy koniunkcje bitową z maską o wartości 1. Na końcu przekształcamy wynik operacji z typu integer na boolean i zwracamy.
	
	\subsubsection{TSelf setBit(int index, boolean value)}
	\begin{listing}[h]
		\begin{minted}{java}
public TSelf setBit(int index, boolean value) {
	int mask = 1 << index;
	int newValue;
	if(value) {
		newValue = this.getUnsignedValue() | mask;
	} else {
		newValue = (this.getUnsignedValue() & ~mask);
	}
	return createNewOfUnsigned(newValue);
}
		\end{minted}
		\caption{Metoda TSelf setBit(int index, boolean value)}
		\label{listing:xbitMethodSetBit}
	\end{listing}	
	Kod \ref{listing:xbitMethodSetBit} prezentuje metodę modyfikujący bit o danym indeksie, o daną wartość, a następnie zwracający nową zmodyfikowany obiekt. Zwracany typ TSelf to typ generyczny reprezentujący docelową liczbę n-bitową. W linii nr 2 tworzymy maskę bitową, wykonując przesunięcie bitowe na liczbie 1 o wartość parametru "index", czyli maska będzie posiadała tylko jeden bit ustawiony na prawdę logiczną, i będzie to bit o indeksie z parametru. W liniach 5 i 7 wykonuje operacje bitowe mające na celu zmianę bitu na 1 lub 0 w zależności od parametru "value". 
	\begin{itemize}  
		\item Dla "value" równego 1 wykonuje się linia 5. Najpierw pobieramy obecną wartość obiektu i wykonujemy alternatywę bitową na niej, i na wcześniej zbudowanej masce. Wynik przypisujemy do zmiennej "newValue" będącej buforem.
		\item Dla "value" równego 10 wykonuje się linia 7. Nową wartość uzyskujemy przez koniunkcje bitową aktualnej wartości oraz negacji maski. 
	\end{itemize} 
	
	\subsubsection{int getValueOfBits(int startIndexBit, int stopIndexBit)}
	\begin{listing}[h]
		\begin{minted}{java}
public int getValueOfBits(int startIndexBit, int stopIndexBit) {
	if(startIndexBit<stopIndexBit) {
		throw new NumberFormatException();
	}

	int buff = valueContainer >>> stopIndexBit;
	return buff & (~(Integer.MAX_VALUE<<(startIndexBit-stopIndexBit+1)));
}
		\end{minted}
		\caption{Metoda int getValueOfBits(int startIndexBit, int stopIndexBit)}
		\label{listing:getValueOfBits}
	\end{listing}
	 Kod \ref{listing:getValueOfBits} przedstawia metodę przyjmującą dwa parametry, które są indeksami dwóch bitów, a zwraca wartość od indeksu pierwszego argumentu, do drugiego. Dla przykładu, jeśli wartość obiektu to binarnie 00001110, argument startIndexBit jest równy 4, a stopIndexBit jest równy 1, to metoda zwróci binarnie wartość 0111.
	 
	 W liniach 1 i dwa sprawdzamy poprawność argumentów i ewentualnie wyrzucamy wyjątek.  Linia 6 tworzy bufor, wykonując przesunięcie bitowe w prawo na wartości obiektu, o wartość drugiego, końcowego indeksu podanego w parametrze metody.
	
	Linia nr 7 wykonuje koniunkcję bitową na buforze, i wyniku operacji przesunięcia bitowego w lewo wartości wypełnionej binarnej jedynkami, o wyniki operacji indeksPoczątku-indeksKońca+1. 
		
	\subsubsection{Inne metody}
	Klasa XBit posiada inne metody, zbyt proste w swojej budowie aby były warte głębszego opisywania. Zostaną one jedynie wymienione poniżej w celu lepszego zobrazowania wszystkich funkcjonalności.
	\begin{itemize}  
		\item public abstract short getSize() - abstrakcyjna klasa zwracająca ilość bitów jakie przechowuje docelowa liczba
		\item public abstract int getMinSignedValue() - zwraca minimalną liczbę ze znakiem jaka może być przechowywana w obiekcie
		\item public abstract int getMaxSignedValue() - zwraca maksymalną liczbę ze znakiem jaka może być przechowywana w obiekcie
		\item public int getMinUnsignedValue() - zwraca minimalną liczbę bez znaku jaka może być przechowywana w obiekcie (czyli zawsze przyjmuje wartość 0)
		\item public abstract int getMaxUnsignedValue() - zwraca maksymalną liczbę bez znaku jaka może być przechowywana w obiekcie
		\item public boolean isNegative() - zwraca wartość bitu o największym indeksie, który w kodowaniu u2 decyduje czy liczba jest mniejsza od zera
		\item public int getSignedValue() - zwraca wartość w kodowaniu U2
		\item public int getUnsignedValue() - zwraca wartość w kodowaniu NKB
	\end{itemize}
	
	
	\subsection{Implementacja klasy Xbit8}
	Klasa XBit8 dziedziczy funkcjonalności po abstrakcyjnej klasie XBit, która została zbudowana w taki sposób aby klasy po niej dziedziczące były jak najmniejsze. Metody które należą do XBit8 i nie zostały odziedziczone nie są na tyle skomplikowane aby warto było je głębiej opisywać, z tego powodu zostaną wymienione poniżej i krótko opisane bez szczegółów implementacji.
	\begin{itemize}  
		\item public static XBit8 valueOfUnsigned(int value) - tworzy nowy obiekt o wartości bez znaku
		\item public static XBit8 valueOfSigned(int value) - tworzy nowy obiekt o wartości ze znakiem
		\item public static XBit8 valueOfBooleanArray(boolean[] values) - tworzy nowy obiekt na podstawie tablicy elementów o typie boolean.
	\end{itemize}
	
	\subsection{Implementacja klasy Xbit16}
	XBit16 to klasa reprezentująca liczbę 16 bitową. Dziedziczy ona po XBit dzięki temu nie musi implementować najbardziej podstawowych funkcji. 
	
	
	\subsubsection{public static XBit16 valueOfHighAndLowInBigEndian(XBit8 high, XBit8 low)}
	\begin{listing}[h]
		\begin{minted}{java}
public static XBit16 valueOfHighAndLowInBigEndian(XBit8 high, XBit8 low) {
	ByteBuffer bb = ByteBuffer.allocate(2);
	bb.order(ByteOrder.BIG_ENDIAN);
	bb.put((byte)high.getSignedValue());
	bb.put((byte)low.getSignedValue());
	return new XBit16(bb.getShort());
}
		\end{minted}
		\caption{Metoda XBit16 valueOfHighAndLowInBigEndian(XBit8 high, XBit8 low)}
		\label{listing:valueOfHighAndLowInBigEndian}
	\end{listing}
	W kodzie \ref{listing:valueOfHighAndLowInBigEndian} pokazano implementację metody "valueOfHighAndLowInBigEndian" tworzącą liczbę 16bitową w formacie zapisu big endian (kolejność bajtów zgodna z "ludzkim" zapisem, najbardziej znaczący bajt umieszczony jest jako pierwszy).  Jako parametry przyjmowane są dwie liczby 8-bitowe o nazwach "high" i "low". 
	Na początku tworzony jest obiekt klasy ByteBuffer należącej do standardowej biblioteki java, a następnie ustawiana jest kolejność w jakiej będą dodawane kolejne bajty (linia nr 3). Następne linie prezentują dodanie do bufora kolejno większego (linia nr 4) i młodszego  bajtu (linia nr 5). Na koniec wykonano metode getShort() zwracająca docelową liczbę i na jej podstawie tworzymy reprezentacje klasy XBit16.
	
	\subsubsection{public static XBit16 valueOfHighAndLowInLittleEndian(XBit8 high, XBit8 low)}
		\begin{listing}[h]
		\begin{minted}{java}
public static XBit16 valueOfHighAndLowInLittleEndian(XBit8 high, XBit8 low) {
	return valueOfHighAndLowInBigEndian(low, high);
}
		\end{minted}
		\caption{Metoda XBit16 valueOfHighAndLowInLittleEndian(XBit8 high, XBit8 low)}
		\label{listing:valueOfHighAndLowInLittleEndian}
	\end{listing}
	W kodzie \ref{listing:valueOfHighAndLowInLittleEndian} zaprezentowano metodę tworzącą liczbę 16-bitową w formacie zapisu little endian, czyli najbardziej znaczący bajt umieszczony jest jako ostatni (odwrotnie do "ludzkiego" zapisu gdzie najbardziej znacząca cyfra jest pierwsza).
	Implementacja metody jest dosyć prosta, wykonuje ona metodę valueOfHighAndLowInBigEndian() z zamienioną kolejnością parametrów.


	\subsubsection{inne metody}
	Poniżej wymieniono metody które są zbyt proste w implementacji aby warto było je opisywać pojedynczo. 
	\begin{itemize}  
		\item public static XBit16 valueOfUnsigned(int value) - tworzy nowy obiekt o wartości bez znaku
		\item public static XBit16 valueOfSigned(int value) - tworzy nowy obiekt o wartości z znakiem
	\end{itemize}


	\subsection{Implementacja klasy XbitUtils}
	XbitUtils to klasa implementująca operacje arytmetyczne i bitowe na obiektach XBit8 oraz XBit16. Posiada ona dwie klasy wewnętrzne, Arithmetic8bitResult oraz Arithmetic16bitResult reprezentujące wynik operacji arytmetycznych, które oprócz samego wyniku powinny informować o wystąpieniu przeniesienia i przepełnienia. Klasy te zaprezentowano w kodach \ref{listing:arithmetic8bitResult} oraz \ref{listing:arithmetic16bitResult}
	
	\begin{listing}[h]
		\begin{minted}{java}
public static class Arithmetic8bitResult {
	public XBit8 result;
	boolean carry = false;
	boolean overflow = false;
}
		\end{minted}
		\caption{Klasa Arithmetic8bitResult}
		\label{listing:arithmetic8bitResult}
	\end{listing}

	\begin{listing}[h]
		\begin{minted}{java}
public static class Arithmetic16bitResult {
	public XBit16 result;
	boolean carry = false;
	boolean overflow = false;
}
		\end{minted}
		\caption{Klasa Arithmetic16bitResult}
		\label{listing:arithmetic16bitResult}
	\end{listing}
	
	\subsubsection{public static XBit8 incrementBy(XBit8 value, int incrementer)}
	\begin{listing}[h]
		\begin{minted}{java}
public static XBit8 incrementBy(XBit8 value, int incrementer) {
	int unsignedValue = value.getUnsignedValue();
	unsignedValue += incrementer;
	unsignedValue = unsignedValue & XBit8.MAX_UNSIGNED_VALUE;
	return XBit8.valueOfUnsigned(unsignedValue);
}
		\end{minted}
		\caption{Metoda XBit8 incrementBy(XBit8 value, int incrementer)}
		\label{listing:incrementBy}
	\end{listing}
	Kod \ref{listing:incrementBy} prezentuje metodę inkrementującą liczbę 8 bitową o daną wartość. Aby wykonać operację, metoda konwertuje parametr "value" na liczbę typu integer (linia nr 2) i wykonuje inkrementacje za pomocą standardowej operacji dodawania w języku java (linia nr 3). Następnie w linii nr 4 wykonuje operacje iloczynu bitowego na uzyskanej nowej wartości, oraz masce bitowej reprezentującej największą możliwą wartość jaką może przechowywać liczba 8-bitowa (celem tej operacji jest nie dopuszczenie do sytuacji, w której wynik inkrementacji nie mieści się na ośmiu bitach). Na koniec tworzona jest i zwracana instancja klasy XBit8. 
	
	Zasada działania metody "XBit16 incrementBy(XBit16 value, int incrementer)" jest podobna, dlatego zostanie pominięta.
		
	\subsection{public static XBit8 negativeOf(XBit8 value)}
	\begin{listing}[h]
		\begin{minted}{java}
public static XBit8 negativeOf(XBit8 value) {
	int buf = (~value.getUnsignedValue()) & XBit8.MAX_UNSIGNED_VALUE;
	return XBitUtils.incrementBy(
		XBit8.valueOfUnsigned(buf),
		1
	);
}
		\end{minted}
		\caption{Metoda XBit8 negativeOf(XBit8 value)}
		\label{listing:negativeOf}
	\end{listing}
	Funkcja tworząca liczbę negatywną do podanej została zaprezentowana w kodzie \ref{listing:negativeOf}. Operacja polega na wykonaniu iloczynu bitowego na negacji bitowej danej wartości, oraz liczbie 255. Następnie wykonany wynik zostaje zainkrementowany. 
		
	Zasada działania metody "XBit16 negativeOf(XBit16 value)" jest identyczna. Jedyną różnicą jest operacja iloczynu bitowego którą wykonujemy nie na liczbie 255 a 65535.
	
	\subsection{public static Arithmetic8bitResult addTwo8bits(XBit8 value1, XBit8 value2)}
	\begin{listing}[h]
		\begin{minted}{java}
public static Arithmetic8bitResult addTwo8bits(XBit8 value1, XBit8 value2) {
	Arithmetic8bitResult ret = new Arithmetic8bitResult();
	int result = value1.getUnsignedValue() + value2.getUnsignedValue();
	if(result>XBit8.MAX_UNSIGNED_VALUE) {
		ret.carry = true;
	}
	ret.result = XBit8.valueOfUnsigned(result & XBit8.MAX_UNSIGNED_VALUE);
	
	ret.overflow =((value1.getBit(7) && value2.getBit(7) && !ret.result.getBit(7)) ||
	(!value1.getBit(7) && !value2.getBit(7) && ret.result.getBit(7)));
	
	return ret;
}
		\end{minted}
		\caption{Metoda Arithmetic8bitResult addTwo8bits(XBit8 value1, XBit8 value2)}
		\label{listing:addTwo8bits}
	\end{listing}
	Kod \ref{listing:addTwo8bits} prezentuje metodę realizującą dodawanie dwóch liczb 8-bitowych. Wynikiem jej działania jest obiekt klasy Arithmetic8bitResult przechowujący wynik operacji, i flagi przepełnienia lub przeniesienia. 
	
	W linii nr 3, metoda wykonuje samą operacje dodawania. Linia nr 4 sprawdza czy nastąpiło przepełnienie, a linia nr 5 ustawia flagę. Linia nr 7 wykonuje "obcięcie" bitów w przypadku, gdy wynik nie będzie w stanie zmieścić się w liczbie 8-bitowej. 
	
	Linia nr 9 i 10 ustawia flagę przepełnienia według trzech zasad \cite{overflowRules}:
	\begin{enumerate}
		\item Jeśli suma dwóch liczb dodatnich daje wynik ujemny, suma została przepełniona. 
		\item Jeśli suma dwóch liczb ujemnych daje wynik dodatni, suma została przepełniona.
		\item W każdym innym przypadku przepełnienie nie wystąpiło.
	\end{enumerate}
	Przyjmujemy że najbardziej znaczący bit jest bitem znaku.
	
	Prezentacja funkcji realizującej dodawanie dwóch liczb 16-bitowych została pominięta, ponieważ odbywa się na tej samej zasadzie.
	
	\subsection{public static Arithmetic8bitResult subTwo8bits(XBit8 value1, XBit8 value2)}
	\begin{listing}[h]
		\begin{minted}{java}
public static Arithmetic8bitResult subTwo8bits(XBit8 value1, XBit8 value2) {
	return addTwo8bits(
		value1,
		negativeOf(value2)
	);
}
		\end{minted}
		\caption{Metoda Arithmetic8bitResult subTwo8bits(XBit8 value1, XBit8 value2)}
		\label{listing:subTwo8bits}
	\end{listing}
	W kodzie nr \ref{listing:subTwo8bits} zaprezentowano metodę realizującą odejmowanie jednej liczby 8-bitowej, od drugiej. Metoda wykorzystuje regułę 
	$ a - b = a+(-b) $ mówiącą, że odejmowanie dwóch liczb można zastąpić dodawaniem, negując drugi składnik odejmowania. 
	Metoda realizująca odejmowanie dwóch liczb 16-bitowych działa w analogiczny sposób.
	
	
	
	
	




	
	
	
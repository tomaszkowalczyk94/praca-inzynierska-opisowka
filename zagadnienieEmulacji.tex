\chapter{Zagadnienie emulacji}
	
	Emulator w kontekście informatyki, oznacza program który jest przystosowany do uruchomienia na specyficznym urządzeniu lub/i systemie, i pozwala na uruchomienie programów napisanych z przeznaczeniem dla innego urządzenia/systemu. \cite{howDoIWriteAnEmulator}
	
	Inną ciekawą definicją emulatora podał Victor Moya del Barrio "Emulacja w informatyce oznacza emulowanie zachowania urządzenia lub oprogramowania za pomocą innego oprogramowania lub urządzenia" %to jest tłumaczenie z ang, powinienem jakoś oznaczyć że to jest tłumaczenie a nie cytat??
	\cite{studyofthetechniquesforemulationprogramming}
		
	Emulacje CPU można przeprowadzić na 3 sposoby:\cite{fms_komkon_org_howto}	
	\begin{itemize}  
		\item emulacja przez interpretowanie
		\item emulacja przez statyczną re-kompilacje
		\item emulacja przez dynamiczną re-kompilacje
	\end{itemize} 
	Każda z tych metod wymaga oddzielnego omówienia.
	
	\section{emulacja przez interpretowanie}
	Interpreter to najprostszy rodzaj emulatora. Odczytuje on w pętli kod programu z wirtualnej pamięci. Odczytany bajt (lub bajty, rozkaz procesora może być wielobajtowy) zawiera informacje o rodzaju operacji jaką CPU powinno wykonać. Interpreter ma za zadanie odkodować informacje o operacji, a następnie ją wykonać. Między kolejnymi rozkazami powinien on zmienić wirtualne parametry (np inkrementacja licznika rozkazów), sprawdzić czy nie zostało wywołane przerwanie, obsłużyć urządzenia wejścia wyjścia, liczniki, kartę graficzną, lub wykonać inne operacje zależne od emulowanego urządzenia. Przykładowa struktura interpretera została przedstawiona w kodzie  \ref{listing:interpreter}
		
	\lstinputlisting[language=C, label={listing:interpreter}, caption={Przykładowa struktura interpretera procesora},captionpos=b]{listings/interpreter.c}
	
	Emulacja przez interpretowanie jest najwolniejszą formą emulacji, ale także najłatwiejszą w debugowania. Pozwala na prześledzenie wykonania operacji, i podgląd wewnętrznych stanów urządzenia. Z tego powodu jest najczęściej wybierana w debuggerach procesorów, mikro-kontrolerów dla programistów.
	
	\section{emulacja przez statyczną re-kompilacje}
	\section{emulacja przez dynamiczną re-kompilacje}
	
	\section{różnica między symulacją a emulacją} 
	% Brakuje mi do tego bibliografii
	% linki z których czerpałem wiedze:
	% https://www.guru99.com/real-device-vs-emulator-testing-ultimate-showdown.html
	% http://fms.komkon.org/EMUL8/HOWTO.html
	% https://stackoverflow.com/questions/1584617/simulator-or-emulator-what-is-the-difference
	% https://www.quora.com/What-are-the-differences-between-simulation-and-emulation
	% TO NAJLEPSZE: https://softwareengineering.stackexchange.com/questions/134746/whats-the-difference-between-simulation-and-emulation
	Zagadnienie emulacji często mylone jest z symulacją. Nie są to jednak jednoznaczne pojęcia.
	
	W książe "Study of the techniques for emulation
	programming" Victor Moya del Barrio podaje taką to definicje emulatora "An emulator tries to duplicate the behaviour of a full computer using software programs in a different computer. "\cite{studyofthetechniquesforemulationprogramming}
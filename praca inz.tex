\documentclass[12pt,a4paper]{report}
\usepackage[utf8]{inputenc}
\usepackage[MeX]{polski}

\usepackage{listings}

\usepackage{xcolor}
\usepackage{minted}
\usepackage{array} 
\usepackage{chngcntr}
\usepackage{changepage}
\usepackage{indentfirst}  %wcięcie w pierwszym akapicie
\usepackage{graphicx}
\usepackage{float}
\usepackage{mathptmx}
\usepackage{titlesec} % zmiana rozmiaru tytułów 
\usepackage{tocloft}

\graphicspath{ {./image/} }

\setminted[java]{
	frame=lines,
	framesep=2mm,
	baselinestretch=1.2,
	fontsize=\footnotesize,
	linenos
}

\renewcommand{\listingscaption}{Kod}

% marginesy
\usepackage[inner=2.5cm, outer=2.5cm, top=2.5cm, bottom=2.5cm, bindingoffset=1cm]{geometry}
\linespread{1.3} %1.3 da interlinie 1.5cm

%puste miejsca w liniach spisu treści wypełnione kropkami
\renewcommand{\cftchapleader}{\cftdotfill{\cftdotsep}}
\renewcommand{\cftsecleader}{\cftdotfill{\cftdotsep}}
\renewcommand{\cftsubsecleader}{\cftdotfill{\cftdotsep}}

%Kropka po numerze sekcji i podsekcji
\titleformat{\chapter}
{\normalfont\Huge\bfseries}{\thechapter.}{20pt}{\Huge}

%Kropka po numerze sekcji i podsekcji.
\makeatletter
\renewcommand\@seccntformat[1]{\csname the#1\endcsname.\quad}
\renewcommand\numberline[1]{#1.\hskip0.7em}
\makeatother
%Tu się kończy definicja kropki po numerze sekcji i podsekcji.


\AtBeginDocument{\counterwithin{listing}{chapter}} % numeracja fragmentów z kodem


%rozmiar dla rozdziałów i tytułów
\titleformat{\chapter}
{\normalfont\large\bfseries\fontsize{14}{0}}{\thechapter.}{1em}{}

\titleformat{\section}
{\normalfont\bfseries\fontfamily{times}\fontsize{12}{0}}{\thesection.}{1em}{}



\begin{document}
	
	\makeatletter
\newcommand{\linia}{\rule[1em]{\linewidth}{0.4mm}}

\def\studentNo#1{\gdef\@studentNo{#1}}
\def\@studentNo{\@latex@warning@no@line{No \noexpandstudentNo given}}

\renewcommand{\maketitle}{\begin{titlepage}

    \begin{center}{\large
	    POLITECHNIKA  ŚWIĘTOKRZYSKA \\
	    Wydział Elektrotechniki, Automatyki i Informatyki}
	\linia
    \end{center}
    \vspace{2.5cm}
   
    \begin{center}
		\normalsize \textsc{\@author} \par
		\normalsize Numer albumu: \@studentNo \par	
		\vspace{2,5cm}
		\LARGE \textsc{\@title}\par
		\normalsize
		\vspace{5mm}
		Praca dyplomowa inżynierska\\
	    na kierunku  Informatyka\\
	\end{center}
	\vfill
	
    \begin{flushright}
	    Opiekun pracy dyplomowej:\\
        dr inż. Arkadiusz Chrobot\\
        Zakład Informatyki
     \end{flushright}
   \vspace{1,5cm}
    \begin{center}
	    Kielce, 2019%\@date
    \end{center}
  \end{titlepage}%
}
\makeatother
	\author{Tomasz Kowalczyk}
	\studentNo{083562}
	\title{Emulator procesora Zilog Z80}
	\maketitle
	
	\leavevmode\thispagestyle{empty}\newpage
	
	\pagestyle{empty}
	\begin{figure}[H]
		\vspace{-2.5cm}
		\hspace{-3.5cm}
		\includegraphics[width=.9\paperwidth]{testa4.png}
	\end{figure}
	
	\leavevmode\thispagestyle{empty}\newpage
	
	\begin{figure}[H]
		\vspace{-2.5cm}
		\hspace{-3.5cm}
		\includegraphics[width=.9\paperwidth]{testa4.png}
	\end{figure}

	\leavevmode\thispagestyle{empty}\newpage
	
	\thispagestyle{empty}
\makeatletter
\begin{center}
	\large \textbf{\@title}
\end{center} 
\makeatother

\subsection*{Streszczenie}
Praca ma na celu stworzenie programu emulującego procesor Zilog Z80. Aplikacja ta została podzielona na trzy niezależne części. Xbit to moduł upraszczający wykonywanie operacji arytmetycznych i~bitowych na liczbach binarnych. Z80emu-core jest częścią aplikacji realizującą proces emulacji. Z80emu-gui implementuje interfejs użytkownika. Aplikacja pozwala wykonać program dla procesora Z80 w~sposób ciągły lub krokowy. Interfejs użytkownika umożliwia modyfikację i~podgląd rejestrów oraz pamięci procesora.
\newline
Słowa kluczowe:
\begin{table}[!h]
\label{kluczowe}
\begin{tabular}{l l}

$\bullet$ Emulator          &  $\bullet$ Java FX \\
$\bullet$ Zilog Z80                              &       $\bullet$ Architektura 8-bitowa \\
$\bullet$ Java 8     &        \\

\end{tabular}
\end{table}

%\selectlanguage{english}
\makeatletter
\begin{center}
	\large \textbf{Zilog Z80 CPU emulator}
\end{center} 
\makeatother

\subsection*{Summary}
The aim of the thesis is to develop an emulator of Zilog Z80 CPU. The application has been divided into three independent parts. Xbit is a module that simplifies arithmetic and bit operations on binary numbers. Z80emu-core is a part of the application that implements the emulation process. Z80emu-gui constitutes the user interface. The application allows the user to perform the program in a continuous or a step-by-step way.
\newline
Keywords:

\begin{table}[!h]
\label{keyword}
\begin{tabular}{l l}

$\bullet$ Emulator          &  $\bullet$ Java FX \\
$\bullet$ Zilog Z80                              &       $\bullet$ 8-bit architecture \\
$\bullet$ Java 8     &        \\

\end{tabular}
\end{table}
	
	\leavevmode\thispagestyle{empty}\newpage
	\leavevmode\thispagestyle{empty}\newpage
	
	%strona ze spisem treści
	\pagebreak
	\pagestyle{plain}
	\setcounter{page}{9} %start numeracji od strony 9
	\tableofcontents	 %generuj spis treści na podstawie \chapter i \section
	
	\leavevmode\newpage %pusta lub kontynuacja spisu treści
	
	\chapter{Wstęp}

	Celem pracy jest wykonanie emulatora procesora Zilog Z80. Aplikacja umożliwia wczytanie programu w postaci kodu maszynowego, deasemblację i wykonanie. Dostępne są dwa tryby wykonania: ciągły i krokowy. W obu przypadkach emulator obrazuje stan rejestrów, jak również umożliwia podgląd i zmianę zawartości pamięci programu. Aplikacja została zaimplementowana w języku Java.
	
	% to jest skopiowane prosto z książki
	Procesor Zilog z80 był szlagierem rynku mikroprocesorowego. \cite{karczmarczuk}
	Został wydany na rynek w roku 1976, i szybko zdominował rynek 8-bitowych procesorów.
	
	Jednym z jego powodów sukcesu na rynku, była prostota w sprzęganiu go z innymi urządzeniami, szczególnie pamięciami. Inną jego zaletą była lista rozkazów zgodna z popularnym w tamtym czasie procesorem, mianowicie Intelem 8086, co umożliwiało uruchamianie programów napisanych z pierwotnym przeznaczeniem dla Intela 8080 na Zilogu Z80. \cite{karczmarczuk}
	
	Urządzenie to mimo zalet, ma również jedną dużą wadę. Jego wewnętrzna budowa była złożona jak na tamte czasy, wyjścia nie były ułożone w logiczny sposób (widoczne na rysunku  \ref{img:z80wyprowadzenia}), a lista rozkazów składała się z 158 pozycji, w tym 78 z nich zgodnych z Intel 8080A \cite{manual}
	
	
	\begin{figure}[h]
		\centering
		\includegraphics[width=0.5\textwidth]{z80wyprowadzenia}
		\caption{Wyprowadzenia mikroprocesora Z80 \cite{karczmarczuk}}
		\label{img:z80wyprowadzenia}
	\end{figure}
			
	\begin{figure}[h]		
		\centering
		\includegraphics[width=0.6\textwidth]{app1}
		\caption{Widok główny emulatora}
	\end{figure}
	
	Samą aplikacje wykonano w języku Java 8 i biblioteki graficznej Java FX. 
	Interfejs użytkownika został podzielony na 3 części: 
	\begin{itemize}  
		\item widok kodu programu napisany w języku asembler, wraz z  wynikowym kodem maszynowym  
		\item widok pamięci w formie tabeli. Aby uzyskać adres odpowiadający danej komórce, należy dodać do siebie wartość \underline{?????}. Edycje wykonujemy przez dwukrotne kliknięcie w komórkę tabeli, wpisaniu nowej wartości i zatwierdzeniu klawiszem Enter.
		\item widok stanu wewnętrznych rejestrów procesora, wraz z przyciskami debugującymi.   
	\end{itemize}
	
	W aplikacji każda wyświetlona wartość podana jest w systemie heksadecymalnym, i także w takiej notacji wprowadzamy wartości (oprócz pola do edycji kodu asemblera, gdzie możemy używać innych notacji).
	
	\chapter{Zagadnienie emulacji}
	
	Emulator w kontekście informatyki, oznacza program który jest przystosowany do uruchomienia na specyficznym urządzeniu lub/i systemie, i pozwala na uruchomienie programów napisanych z przeznaczeniem dla innego urządzenia/systemu\cite{howDoIWriteAnEmulator}. 
	
	Inną ciekawą definicję emulatora podał Victor Moya del Barrio "Emulacja w informatyce oznacza emulowanie zachowania urządzenia lub oprogramowania za pomocą innego oprogramowania lub urządzenia"
	\cite{studyofthetechniquesforemulationprogramming}.
		
	Emulacje CPU można przeprowadzić na 3 sposoby:\cite{fms_komkon_org_howto}	
	\begin{itemize}  
		\item emulacja przez interpretowanie
		\item emulacja przez statyczną re-kompilacje
		\item emulacja przez dynamiczną re-kompilacje
	\end{itemize} 
	Każda z tych metod wymaga oddzielnego omówienia.
	
	\section{Emulacja przez interpretowanie}
	Interpreter to najprostszy rodzaj emulatora. Odczytuje w pętli kod programu z wirtualnej pamięci. Odczytany bajt (lub bajty, rozkaz procesora może być wielobajtowy) zawiera informacje o rodzaju operacji jaką CPU powinno wykonać. Interpreter ma za zadanie odkodować informacje o operacji, a następnie ją wykonać. Między kolejnymi rozkazami powinien on zmienić wirtualne parametry (np inkrementacja licznika rozkazów), sprawdzić czy nie zostało wywołane przerwanie, obsłużyć urządzenia wejścia/wyjścia, liczniki, kartę graficzną, lub wykonać inne operacje zależne od emulowanego urządzenia. Przykładowa struktura interpretera została przedstawiona w kodzie \ref{listing:interpreter}.
	
	\begin{listing}[h]
		\inputminted{java}{listings/interpreter.c}
		\caption{Przykładowa struktura interpretera procesora}
		\label{listing:interpreter}
	\end{listing}
		
	Emulacja przez interpretowanie jest najwolniejszą formą emulacji, ale także najłatwiejszą w debugowaniu. Pozwala na prześledzenie wykonania operacji, i podgląd wewnętrznych stanów urządzenia. Z tego powodu jest najczęściej wybierana w debuggerach procesorów, mikro-kontrolerów dla programistów.
	
	\section{Statyczna re-kompilacja}
	Statyczna re-kompilacja (ang. "Static binary translation") to proces konwertowania kodu maszynowego na inny kod maszynowy przeznaczony dla docelowej architektury. Plik wykonywalny tłumaczony jest raz, za jednym podejściem przez cały plik. Problemem tego rozwiązania są instrukcje skoków pośrednich czyli takich gdzie adres skoku przechowywany jest w rejestrze lub pamięci, i może on być uzyskany tylko podczas wykonywania programu. W takim przypadku niemożliwym jest przetłumaczenie wszystkich instrukcji pliku wykonywalnego\cite{uqbt}. 
	
	
	\section{Dynamiczna re-kompilacja}
	Dynamiczna re-kompilacja (ang. "Dynamic binary translator"), w odróżnieniu od translacji dynamicznej, tłumaczy kod blokami, podczas jego wykonywania. Re-kompilacja występuje "na żądanie" co jest wolniejsze od statycznej re-kompilacji, ale rozwiązuje problem związany z statycznym tłumaczeniem kodu wykonywanego za instrukcjami skoków pośrednich.
	
	Raz przetłumaczony fragment kodu jest przechowywany w pamięci, na wypadek jego ponownego użycia, co pozwala zoptymalizować ten sposób emulacji\cite{uqbt}.
	
	
	Dynamicznej rekompilacji używa w dużym stopniu maszyna wirtualna javy. Wczesne wersje JVM (Java Virtual Machin) używały do swojego działania interpreterów, co okazało się mało wydajne. Dobrym sposobem na optymalizację maszyny wirtualnej okazało się dynamiczne tłumaczenie kodu maszynowego\cite{dynamicRecompilationInJava}. 
	
	
	
	\section{Różnica między symulacją a emulacją} 
	Różnicę w emulacji a symulacji obrazuje tabela \ref{table:emulationAndSimulation}.
	
	\begin{table}
		\centering
		\begin{tabular}{ m{7cm} | m{7cm} }
			Symulator &  Emulator  \\ 
			\hline
			System zdolny do naśladowania innego systemu w pewnym stopniu  & System który naśladuje dokładne zachowanie innego systemu \\   
			\hline
			Może nie przestrzegać wszystkich reguł symulowanego systemu & Ściśle przestrzega parametrów i reguł emulowanego systemu \\ 
			\hline
			Modeluje aplikacje i zdarzenia & Kopiuje zachowanie systemów \\ 
		\end{tabular}
		\caption{Tabela prezentująca różnice między symulacją a emulacją \cite{emulationOrSimulation}}
		\label{table:emulationAndSimulation}
	\end{table}
	
	W kontekście informatyki, symulator to program komputerowy, który modeluje zachowania i funkcje innego realnego systemu lub zjawiska (np. prowadzenie pojazdu). Nie jest wymagane, aby potrafił odwzorować jego wszystkie zachowania i funkcje. Symulator nie będzie w stanie wykonywać realnych zadań emulowanego urządzenia i nie będzie w stanie go zastąpić.
	
	Natomiast emulator ma za zadanie "udawać" dane urządzenie/zjawisko w takim stopniu, i na takim poziomie, aby był w stanie zastąpić emulowane urządzenie i funkcjonować tak jak one.
	
	% linki z których czerpałem wiedze:
	% https://www.guru99.com/real-device-vs-emulator-testing-ultimate-showdown.html
	% http://fms.komkon.org/EMUL8/HOWTO.html https://stackoverflow.com/questions/1584617/simulator-or-emulator-what-is-the-difference
	% https://www.quora.com/What-are-the-differences-between-simulation-and-emulation
	% TO NAJLEPSZE: https://softwareengineering.stackexchange.com/questions/134746/whats-the-difference-between-simulation-and-emulation
		
	% W książe "Study of the techniques for emulation programming" Victor Moya del Barrio podaje taką to definicje emulatora "An emulator tries to duplicate the behaviour of a full computer using software programs in a different computer."\cite{studyofthetechniquesforemulationprogramming}. 
	\chapter{Przegląd istniejących rozwiązań}
	Procesor Zilog Z80 dorobił się wielu emulatorów, pisanych kiedyś przez duże firmy, a aktualnie przez hobbystów. Na popularnej usłudze hostingowej Github przeznaczonej dla projektów programistycznych, można znaleźć około 200 repozytoriów z projektami emulującymi Z80, lub emulującymi urządzenia używające tego procesora, co dowodzi jego popularności.\cite{githubZ80Emulators}   
	
	 Poniżej prezentuje najciekawsze pozycje tych programów, które pozwalają na wgląd w  wewnętrzne stany procesora. Przedstawię zarówno komercyjne rozwiązania, jak i te pisane przez amatorów.
	
	\section{Z80 SIMULATOR IDE}
	Dostępny pod adresem http://www.oshonsoft.com/z80.html płatny symulator posiadający najbardziej rozbudowany interfejs z wszystkich wymienionych pozycji. Pozwala on na prezentowanie wewnętrznych stanów procesora, manipulacją przerwaniami, edytor pamięci umożliwiający działający również podczas symulacji, podgląd i manipulacja portami wejścia/wyjścia. Posiada również funkcje typowe dla debuggerów, możliwość wstrzymania działania programu w określonym miejscu, tryb pracy krokowej, interaktywny edytor i kompilator kodu asemblera.\cite{oshonsoftEmulator} Część funkcji została zaprezentowana na rysunku \ref{img:oshonsoftEmulator} 
	
	\begin{figure}[h]
		\centering
		\includegraphics[width=0.7\textwidth]{oshonsoftEmulator}
		\caption{Z80 SIMULATOR IDE \cite{oshonsoftEmulator}}
		\label{img:oshonsoftEmulator}
	\end{figure}
	
	Do wad emulatora należy interfejs, który nie jest intuicyjny. Dla przykładu, w żadnym miejscu nie znajdziemy informacji, o tym w jakim formacie powinny być wprowadzane wartości liczbowe. Brak w programie systemu pomocy, opisów, co może odstraszyć początkującego użytkownika. Dodatkowo jest to rozwiązanie płatne i przeznaczone tylko dla platformy MS Windows. Jest to narzędzie głównie dla specjalistów.
	
	
	\section{ZEMU - Z80 Emulator Joe Moore}
	
	

	Żadne istniejące rozwiązanie nie pozwala na podejrzenie wewnętrznych magistrali procesora
	
	\section{Motywacja}
	dsadsadsa
	\chapter{Projekt aplikacji}
	uml-e, mvc. wzorce projektowe, może maven????
	
	\section{Podział aplikacji}
	Aplikację podzielono na 3 mniejsze moduły, Xbit, z80emu-core, z80emu-gui z czego każdy z nich jest osobnym mniejszym projektem. 
	
	\subsection{XBit}
	Java jest językiem programowania wysokiego poziomu, kompilowanym do kodu bajtowego. Z tego powodu nie jest on zazwyczaj stosowany w emulacji, gdyż kod emulatora musi być uruchamiany w maszynie wirtualnej, co nie należy do optymalnych rozwiązań.
	
	Innym poważnym problemem Javy jest brak typów danych przechowujących tylko wartości dodatnie. James Gosling, jeden z twórców Javy tak argumentuje ich brak: 
	,,Dla mnie jako projektant języków programowania, do których tak naprawdę ostatnimi czasy się nie zaliczam, coś prostego oznaczało skończenie na założeniu że losowy deweloper będzie w stanie zapamiętać specyfikacje. 
	Ta definicja mówi, że na przykład Java i wiele innych języków nie są proste, i w rzeczywistości wiele języków kończy z funkcjonalnościami których do końca nikt nie rozumie. Spytaj jakiegoś programistę języka C o dodatnie typy liczbowe, i odkryjesz że prawie żaden programista C faktycznie nie rozumie co dzieje się z typami bez znaku, czym jest arytmetyka liczb całkowitych. Takie rzeczy sprawiły że C jest językiem skomplikowanym. Myślę że w tej kwestii Java jest prosta"\cite{javaGoslingInterview}.
	
	
	Problem ten rozwiązano tworząc własną bibliotekę o roboczej nazwie XBit. Przechowuje ona liczby 8 i 16 bitowe, które mogą być interpretowane zarówno jako wartości całkowite jak i liczby w kodzie uzupełnień do dwóch. Biblioteka potrafi zwrócić konkretne bity, stworzyć reprezentacje liczby z pojedynczych bitów i typów prymitywnych, obudowuje operacje arytmetyczne z uwzględnieniem przepełnienia i przeniesienia. Kod projektu z jej użyciem staje się czytelniejszy i łatwiejszy do ewentualnej refaktoryzacji. 
	
	\subsection{z80emu-core}
	
	\subsection{z80emu-gui}
	
	
	
	\chapter{Implementacja}
	fragmenty kodu, podział na pakiety	
	\chapter{Testy}
		
	\section{????}
	Bardzo ważną kwestią w projekcie było dokładne pokrycie kodu aplikacji w testach jednostkowych. Emulator mikro-kontrolera to specyficzna aplikacja. Z pozoru mało znaczący błąd może sprawić że emulator stanie się bezużyteczny. 
	
	Dla przykładu, jeśli dla 3 bajtowego rozkazu procesora zwiększymy rejestr PC o 2 zamiast o 3, to nie wykona się następna instrukcja przewidziana przez programistę. Dalsza praca emulatora stanie się nieprzewidywalna, a następna instrukcja całkowicie “wykolei” nasz program który zacznie wykonywać losowe instrukcje. 
	
	Dlatego poprawne wykonanie każdego rozkazu jest tak ważne dla mojego projektu. Aby uchronić się przed tego typu prostymi błędami każdy emulowany rozkaz posiada swój test/testy jednostkowe napisane przy pomocy biblioteki Junit. 
	
	Przykładowy test dla rozkazu LD A, I. Rozkaz ten ładuje zawartość rejestru A z I:
	\lstinputlisting[language=Java]{listings/exampleTest.java}
	
	Opisany przykład ukazuje że prosta z pozoru operacja jak pobranie wartości jednego rejestru i przeniesienie go do innego, wymaga objętościowych testów. Oprócz testowania czy poprawna wartość znajduje się w rejestrze docelowym, musimy sprawdzić także czy flagi CPU zostały ustawione na poprawnych wartościach, czy ilość przewidywanych cykli zegara została poprawnie zwiększona, czy rejestr PC został zainkrementowany. 
	
	\section{Test-driven development}
	TDD to metoda pisania oprogramowania. Zakłada ona że test jednostkowy dla danej funkcjonalności powstaje jako pierwszy. Dopiero po napisaniu testu implementujemy kod programu, a następnie testujemy za pomocą już napisanych testów. Za pomocą TDD była pisana cała aplikacja	
	\chapter{Uwagi i wnioski}
	
	Z  wymienionych celów, nie zrealizowałem jedynie emulacji wewnętrznych magistrali procesora. Nie posiadając dokumentacji technicznych opisujących wewnętrzną budowę mikroprocesora, jedyną opcją było by poddanie urządzenia inżynierii wstecznej, \underline{co już nie jest tematem tej pracy.}
	
	
	
	
	
	
	\begin{thebibliography}{9}
		\bibitem{studyofthetechniquesforemulationprogramming}
		Victor Moya del Barrio
		\emph{Study of the techniques for emulation programming}.
		2001
		
		\bibitem{emulationOrSimulation}
		https://difference.guru/difference-between-simulator-and-emulator/
		
		\bibitem{fms_komkon_org_howto}
		http://fms.komkon.org/EMUL8/HOWTO.html
		
		\bibitem{karczmarczuk}
		Mikroprocesor Z80 Jerzy Karczmarczuk
		
		\bibitem{manual} 
		Oficjalny manual
		
		\bibitem{howDoIWriteAnEmulator} https://www.atarihq.com/danb/files/emu\_vol1.txt
		How do I write an emulator? Daniel Boris, 1999
		
		\bibitem{uqbt} 
		https://www.researchgate.net/publication/239665973\_The\_university\_of\_queensland\_binary\_translator\_uqbt\_framework
		% http://citeseerx.ist.psu.edu/viewdoc/download?doi=10.1.1.87.4982&rep=rep1&type=pdf
		
		\bibitem{dynamicRecompilationInJava}
		https://www.ibm.com/developerworks/library/j-jtp12214/index.html
		
		\bibitem{oshonsoftEmulator}
		http://www.oshonsoft.com/z80.html
		
		\bibitem{githubZ80Emulators}
		https://github.com/search?q=emulator+z80\&type=Repositories
		
		\bibitem{cpm}
		http://www.retrotechnology.com/dri/howto\_cpm.html
		
		\bibitem{zemuImg}
		https://www.retrobrewcomputers.org/n8vem-gg-archive/html-2012/Jul/msg00238.html

		\bibitem{zim}
		http://www.natmac.net/zim/
		
		\bibitem{zimPurpose}
		http://www.natmac.net/zim/manual/index.html
		
		\bibitem{zimImg}
		http://www.natmac.net/zim/GUI6.jpg
		
		\bibitem{javaGoslingInterview}
		http://www.gotw.ca/publications/c\_family\_interview.htm
		
		\bibitem{eduinf}
		https://eduinf.waw.pl/
		
		\bibitem{overflowRules}
		http://sandbox.mc.edu/~bennet/cs110/tc/orules.html
				
	\end{thebibliography}
	
\end{document}
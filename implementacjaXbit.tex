\section{XBit}

\emph{XBit} to moduł mający za zadanie ułatwić operacje bitowe w języku Java, która interpretuje liczby całkowite w kodzie dopełnień do dwóch i nie jest możliwa interpretacja ich w naturalnym kodzie binarnym (w skrócie NKB). Z tego powodu powstał moduł \emph{XBit} implementujący taką operację. Potrafi on odczytać liczbę n{\dywiz}bitową i zwrócić jej wartość jako zmienną typu \emph{int}, interpretując ją jako bity w U2 lub NKB.


Wewnątrz obiektów reprezentujących liczby binarne, wartość przechowywana jest w~polu typu \emph{int}, które nazwano \emph{valueContainer}. Może ono przyjąć teoretycznie wartości od -2147483648 do 2147483647. Uznano jednak, że atrybut ten przechowywał będzie wartość, jaką reprezentują przechowywane bity w notacji NKB (czyli tylko wartości dodatnie). Z tego powodu wartość \emph{valueContainer} nigdy nie będzie ujemna. \newline
Zakres wartości tego pola to od 0 do $ 2^{n}-1, n<32 $ gdzie \emph{n} oznacza liczbę bitów, jaką przechowuje dany obiekt. Ponieważ typ \emph{int} przechowywany jest w czterech bajtach i interpretowany jest przez język Java w notacji U2, maksymalnie może przechować wartość 2 147 483 647. Dla wartości interpretowanej w NKB na czterech bajtach jest to \newline $2^{32}-1 =$ 4 294 967 296. Jest to więcej niż typ \emph{int} może zawierać, dlatego maksymalna możliwa binarna reprezentacja, jaka może zostać zapisana w podobny sposób, to liczba przechowywana na 31 bitach.

Jeśli obiekt \emph{XBit8} będzie przechowywał liczbę binarną 11111111 (czyli odczytując w~NKB to decymalne 255, a w U2 decymalnie -1). Wartość jego pola \emph{valueContainer} będzie wynosić w~kodzie binarnym
00000000 00000000 00000000 11111111. (czyli odczytując zarówno w NKB, jak i w U2 będzie to 255). Niestety nadmiarowość bitów jest w tym rozwiązaniu wymagana, ponieważ liczba w kodzie U2, której najbardziej znaczący bit ma wartość $1$, jest ujemna. 

Podsumowując, \emph{XBit} wykorzystuje fakt takiego samego zapisu liczb dodatnich w U2 i NKB, jeśli tylko będą one zapisane na większej liczbie bitów niż wymagana.


\subsection{Implementacja klasy \emph{XBit}}
\emph{XBit} to klasa abstrakcyjna zawierająca wspólne elementy dla klas \emph{XBit8} i \emph{XBit16} oraz, ewentualnych klas dziedziczących.W~tym podrozdziale zawarto opis najbardziej istotnych metod tej klasy.

\subsubsection{boolean getBit(int index)}

\begin{listing}[h]
	\inputminted{java}{listings/xbit/xbitMethodGetBit.java}
	\caption{Metoda boolean getBit(int index)}
	\label{listing:xbitMethodGetBit}
\end{listing}
Na listingu \ref{listing:xbitMethodGetBit} zaprezentowano metodę \emph{getBit} zwracającą wartość bitu na pozycji o~numerze przekazanym w parametrze. W linii nr 2 sprawdzana jest poprawność parametru \emph{index}. Jeśli jest on niepoprawny, w linii nr 3 wyrzucany jest wyjątek \emph{NumberFormatException}. W linii nr 5 metoda wykonuje przesunięcie bitowe pola \emph{valueContainer} o~podany index. Ponieważ wynikiem jest liczba o typie \emph{int}, wykonywana zostaje koniunkcja bitowa z maską o wartości 1. Na końcu wynik operacji jest przekształcany z typu \emph{int} na \emph{boolean} i zwracany.


\subsubsection{setBit(int index, boolean value)}
\begin{listing}[h]
	\inputminted{java}{listings/xbit/xbitMethodSetBit.java}
	\caption{Metoda TSelf setBit(int index, boolean value)}
	\label{listing:xbitMethodSetBit}
\end{listing}	
Listing \ref{listing:xbitMethodSetBit} prezentuje metodę nadającą bitowi o określonej pozycji, zadaną wartość, a~następnie zwracającą nowy zmodyfikowany obiekt. Typ \emph{TSelf} zwracanej wartości to typ generyczny reprezentujący docelową liczbę n-bitową. W linii nr 2 metoda tworzy maskę bitową, wykonując przesunięcie bitowe na liczbie 1 o wartość parametru \emph{index}, czyli maska będzie posiadała tylko jeden bit ustawiony na wartość $1$, i będzie to bit o~pozycji z~parametru. W liniach 5 i 7 metoda wykonuje operacje bitowe mające na celu zmianę bitu na 1 lub 0 w zależności od parametru \emph{value}:
\begin{itemize}  
	\item dla \emph{value} równego 1 wykonywane są instrukcje zawarte w~linii nr~5. Najpierw pobierana jest obecna wartość obiektu i wykonana alternatywa bitowa  na niej i na wcześniej zbudowanej masce. Metoda przypisuje wynik do zmiennej \emph{newValue} będącej buforem,
	\item dla \emph{value} równego 0 metoda wykonuje instrukcje zawartą w linii nr~7. Nowa wartość uzyskiwana jest przez koniunkcje bitową aktualnej wartości oraz negacji maski. 
\end{itemize} 

\subsubsection{int getValueOfBits(int startIndexBit, int stopIndexBit)}
\begin{listing}[h]
	\inputminted{java}{listings/xbit/getValueOfBits.java}
	\caption{Metoda int getValueOfBits(int startIndexBit, int stopIndexBit)}
	\label{listing:getValueOfBits}
\end{listing}
Listing \ref{listing:getValueOfBits} przedstawia metodę zwracającą fragment wartości binarnej, od pozycji określonej przez argument \emph{startIndexBit}. do \emph{stopIndexBit}, włącznie.

W liniach 2 i 3 sprawdzana jest poprawność argumentów i ewentualnie zostaje wyrzucony wyjątek.  W linii 6 tworzony jest bufor, wykonując przesunięcie bitowe w prawo na wartości obiektu, o wartość parametru \emph{stopIndexBit}. Dzięki tej operacji pozbyto się zbędnych bitów z wartości obiektu po prawej stronie. Następnym krokiem jest pozbycie się bitów po lewej stronie. W tym celu, w linii nr 7 wykonywana jest maska bitowa wypełniona taką liczbą jedynek, jak liczba bitów docelowej wartości. Aby ją wykonać, wykonujemy przesunięcie bitowe w lewo na maksymalnej liczbie, jaką może przyjąć typ \emph{int} (w zapisie binarnym 01111111 11111111 11111111 11111111) o liczbę bitów wartości docelowej - 1. Następnie wykonano negację bitową. Bitów po lewej stronie pozbyto się za pomocą operacji iloczynu bitowego na uzyskanej masce i buforze.

\subsubsection{Inne metody}
Klasa \emph{XBit} posiada inne metody, zbyt proste w swojej budowie, aby były warte dokładniejszego opisywania. Zostaną one jedynie wymienione w celu lepszego zobrazowania wszystkich możliwości.
\begin{itemize}  
	\item \emph{public abstract short getSize()} - abstrakcyjna metoda zwracająca liczbę bitów jakie przechowuje docelowa liczba,
	\item \emph{public abstract int getMinSignedValue()} - zwraca minimalną liczbę ze znakiem jaka może być przechowywana w obiekcie,
	\item \emph{public abstract int getMaxSignedValue()} - zwraca maksymalną liczbę ze znakiem jaka może być przechowywana w obiekcie,
	\item \emph{public int getMinUnsignedValue()} - zwraca minimalną liczbę bez znaku jaka może być przechowywana w obiekcie (czyli zawsze przyjmuje wartość 0),
	\item \emph{public abstract int getMaxUnsignedValue()} - zwraca maksymalną liczbę bez znaku jaka może być przechowywana w obiekcie,
	\item \emph{public boolean isNegative()} - zwraca wartość bitu o największej pozycji, który w kodowaniu U2 decyduje czy liczba jest mniejsza od zera,
	\item \emph{public int getSignedValue()} - zwraca wartość w kodowaniu U2,
	\item \emph{public int getUnsignedValue()} - zwraca wartość w kodowaniu NKB.
\end{itemize}

\subsection{Implementacja klasy \emph{XBit8}}
Klasa \emph{XBit8} dziedziczy po abstrakcyjnej klasie \emph{XBit}, która została zbudowana w taki sposób, aby klasy po niej dziedziczące były jak najmniejsze. Metody, które należą do \emph{XBit8} i nie zostały odziedziczone. Nie są one na tyle skomplikowane, aby warto było je bardziej szczegółowo opisywać, z tego powodu zostaną wymienione i krótko opisane.
\begin{itemize}  
	\item \emph{public static XBit8 valueOfUnsigned(int value)} - tworzy nowy obiekt o wartości bez znaku,
	\item \emph{public static XBit8 valueOfSigned(int value)} - tworzy nowy obiekt o wartości ze znakiem,
	\item \emph{public static XBit8 valueOfBooleanArray(boolean[] values)} - tworzy nowy obiekt na podstawie tablicy elementów o typie \emph{boolean}.
\end{itemize}

\subsection{Implementacja klasy XBit16}
\emph{XBit16} to klasa reprezentująca liczbę 16 bitową. Dziedziczy ona po \emph{XBit}. Dzięki temu nie musi implementować najbardziej podstawowych metod. 


\subsubsection{public static XBit16 valueOfHighAndLowInBigEndian(XBit8 high, XBit8 low)}
\begin{listing}[h]
	\inputminted{java}{listings/xbit/valueOfHighAndLowInBigEndian.java}
	\caption{Metoda XBit16 valueOfHighAndLowInBigEndian(XBit8 high, XBit8 low)}
	\label{listing:valueOfHighAndLowInBigEndian}
\end{listing}
Listingu \ref{listing:valueOfHighAndLowInBigEndian} przedstawia implementację metody \emph{valueOfHighAndLowInBigEndian} tworzącej liczbę 16{\dywiz}bitową w formacie zapisu \emph{big endian} (najbardziej znaczący bajt umieszczony jest jako pierwszy). Przyjmuje ona jako parametry dwie liczby 8{\dywiz}bitowe o nazwach \emph{high} i \emph{low}. 
Metoda tworzy obiekt klasy \emph{ByteBuffer} należącej do standardowej biblioteki języka Java i konfiguruje ją, aby przechowywała dane w formacie \emph{big endian} (linia nr 3). 
Następne linie prezentują operacje dodania do bufora kolejno starszego (linia nr 4) i~młodszego  bajtu (linia nr 5). Na koniec wykonano metodę \emph{getShort()} zwracającą docelową liczbę i na jej podstawie tworzona jest reprezentacja klasy \emph{XBit16}.

\subsubsection{public static XBit16 valueOfHighAndLowInLittleEndian(XBit8 high, XBit8 low)}
\begin{listing}[h]
	\inputminted{java}{listings/xbit/valueOfHighAndLowInLittleEndian.java}
	\caption{Metoda XBit16 valueOfHighAndLowInLittleEndian(XBit8 high, XBit8 low)}
	\label{listing:valueOfHighAndLowInLittleEndian}
\end{listing}
W listingu \ref{listing:valueOfHighAndLowInLittleEndian} zaprezentowano metodę tworzącą liczbę 16{\dywiz}bitową w formacie zapisu \emph{little endian}, czyli najbardziej znaczący bajt umieszczony jest jako ostatni.
Implementacja metody jest dosyć prosta, wykonuje ona metodę \emph{valueOfHighAndLowInBigEndian()} z~zamienioną kolejnością parametrów.


\subsubsection{Inne metody}
Klasa \emph{XBit16} zawiera także metody, które są zbyt proste w implementacji aby warto było je opisywać pojedynczo:
\begin{itemize}  
	\item \emph{public static XBit16 valueOfUnsigned(int value)} - tworzy nowy obiekt o wartości bez znaku,
	\item \emph{public static XBit16 valueOfSigned(int value)} - tworzy nowy obiekt o wartości ze znakiem.
\end{itemize}


\subsection{Implementacja klasy XbitUtils}
\emph{XbitUtils} to klasa implementująca operacje arytmetyczne i bitowe na obiektach \emph{XBit8} i~\emph{XBit16}. Posiada ona dwie klasy wewnętrzne, \emph{Arithmetic8bitResult}  oraz \newline\emph{Arithmetic16bitResult} reprezentujące wynik operacji arytmetycznych i zawierające informacje o wystąpieniu przeniesienia i przepełnienia. Klasy te zaprezentowano w listingach \ref{listing:arithmetic8bitResult} oraz \ref{listing:arithmetic16bitResult}.

\begin{listing}[h]
	\inputminted{java}{listings/xbit/arithmetic8bitResult.java}
	\caption{Klasa Arithmetic8bitResult}
	\label{listing:arithmetic8bitResult}
\end{listing}

\begin{listing}[h]
	\inputminted{java}{listings/xbit/arithmetic16bitResult.java}
	\caption{Klasa Arithmetic16bitResult}
	\label{listing:arithmetic16bitResult}
\end{listing}

\subsubsection{public static XBit8 incrementBy(XBit8 value, int incrementer)}
\begin{listing}[h]
	\inputminted{java}{listings/xbit/incrementBy.java}
	\caption{Metoda XBit8 incrementBy(XBit8 value, int incrementer)}
	\label{listing:incrementBy}
\end{listing}
Listing \ref{listing:incrementBy} prezentuje metodę zwiększającą liczbę 8{\dywiz}bitową o daną wartość. Aby wykonać operację, metoda konwertuje parametr \emph{value} na liczbę typu \emph{int} (linia nr 2) i dodaje wartość parametru \emph{incrementer} (linia nr 3). Następnie w linii nr 4 wykonuje operacje iloczynu bitowego na uzyskanej nowej wartości oraz masce bitowej reprezentującej największą możliwą wartość, jaką może przechowywać liczba 8{\dywiz}bitowa (celem tej operacji jest niedopuszczenie do sytuacji, w której wynik operacji nie mieści się na ośmiu bitach). Na koniec tworzona jest i zwracana instancja klasy \emph{XBit8}. 

Zasada działania metody \emph{XBit16 incrementBy(XBit16 value, int incrementer)} jest podobna, dlatego jej opis zostanie pominięty. 

\subsection{public static XBit8 negativeOf(XBit8 value)}
\begin{listing}[h]
	\inputminted{java}{listings/xbit/negativeOf.java}
	\caption{Metoda XBit8 negativeOf(XBit8 value)}
	\label{listing:negativeOf}
\end{listing}
Funkcja tworząca liczbę o~przeciwnym znaku do podanej została zaprezentowana w~listingu \ref{listing:negativeOf}. Operacja ta polega na wykonaniu iloczynu bitowego na negacji bitowej danej wartości oraz liczbie 255. Ma to na celu niedopuszczenie do sytuacji, w której wartość bufora będzie większa niż 255, a więc nie będzie mogła się zmieścić na ośmiu bitach.
Następnie wykonany wynik zostaje zwiększony o jeden. 

Zasada działania metody \emph{XBit16 negativeOf(XBit16 value)} jest taka sama. Jedyną różnicą jest operacja iloczynu bitowego, którą wykonujemy nie na liczbie 255 a 65535.

\subsection{public static Arithmetic8bitResult addTwo8bits(XBit8 value1, XBit8 value2)}
\begin{listing}[h]
	\inputminted{java}{listings/xbit/addTwo8bits.java}
	\caption{Metoda Arithmetic8bitResult addTwo8bits(XBit8 value1, XBit8 value2)}
	\label{listing:addTwo8bits}
\end{listing}
Listing \ref{listing:addTwo8bits} prezentuje metodę realizującą dodawanie dwóch liczb 8-bitowych. Wynikiem jej działania jest obiekt klasy \emph{Arithmetic8bitResult} przechowujący wynik operacji i flagi przepełnienia lub przeniesienia. 

W linii nr 3 metoda wykonuje operacje dodawania. W linii nr 4 umieszczono sprawdzenie, czy nastąpiło przepełnienie, a linia nr 5 ustawia flagę. Linia nr 7 wykonuje ,,obcięcie" bitów w przypadku, gdy wynik nie będzie w stanie zmieścić się w liczbie 8{\dywiz}bitowej.

Linie 9 i 10 zawierają operację ustawiającą flagę przepełnienia według\newline trzech zasad \cite{overflowRules}:
\begin{enumerate}
	\item jeśli suma dwóch liczb dodatnich daje wynik ujemny, to znaczy, że wystąpiło przepełnienie,
	\item jeśli suma dwóch liczb ujemnych daje wynik dodatni, to również oznaca wystąpienie przepełnienia,
	\item w każdym innym przypadku przepełnienie nie wystąpiło.
\end{enumerate}
Zakłada się, że najbardziej znaczący bit jest bitem znaku.

Prezentacja metody realizującej dodawanie dwóch liczb 16{\dywiz}bitowych została pominięta, ponieważ operacja ta jest przeprowadzana na tej samej zasadzie co 8{\dywiz}bitowa.

\subsection{public static Arithmetic8bitResult subTwo8bits(XBit8 value1, XBit8 value2)}
\begin{listing}[h]
	\inputminted{java}{listings/xbit/subTwo8bits.java}
	\caption{Metoda Arithmetic8bitResult subTwo8bits(XBit8 value1, XBit8 value2)}
	\label{listing:subTwo8bits}
\end{listing}
W listingu \ref{listing:subTwo8bits} zaprezentowano metodę realizującą odejmowanie jednej liczby 8{\dywiz}bitowej, od drugiej. Metoda wykorzystuje regułę 
$ a - b = a+(-b) $ według której odejmowanie dwóch liczb można zastąpić dodawaniem, negując drugi składnik odejmowania. 
Metoda realizująca odejmowanie dwóch liczb 16{\dywiz}bitowych działa w analogiczny sposób.

\subsection{public static XBit8 not8bit(XBit8 value)}
\begin{listing}[h]
	\inputminted{java}{listings/xbit/not8bit.java}
	\caption{Metoda XBit8 not8bit(XBit8 value)}
	\label{listing:not8bit}
\end{listing}
Metoda \emph{not8bit(XBit8 value)} zaprezentowana w listingu \ref{listing:not8bit} wykonuje negację bitową na liczbie 8{\dywiz}bitowej. Operacja wykonywana jest za pomocą standardowego operatora negacji bitowej języka Java (znak tyldy górnej ,,\~{}"). Dodatkową wykonywaną operacją jest iloczyn bitowy z liczbą 255, mający za zadanie wyzerowanie bitów starszych od ósmego. Wersja metody dla liczb 16{\dywiz}bitowych jest analogiczna.

\subsection{public static XBit8 and8bit(XBit8 value1, XBit8 value2)}
\begin{listing}[h]
	\inputminted{java}{listings/xbit/and8bit.java}
	\caption{Metoda XBit8 and8bit(XBit8 value1, XBit8 value2)}
	\label{listing:and8bit}
\end{listing}
Listing \ref{listing:and8bit} prezentuje metodę wykonującą operację sumy logicznej dwóch liczbe \mbox{8{\dywiz}bitowych} (linia nr 3). Linia nr 4 zawiera operację wyzerowania bitów starszych od ósmego dla wyniku, aby mieścił się on w ośmiu bitach. Wersja metody dla liczb \mbox{16{\dywiz}bitowych} działa analogicznie.

\subsection{Metody wykonujące sumę bitową i różnice symetryczną}
Implementacje metod:
\begin{itemize}  
	\item \emph{public static XBit8 or8bit(XBit8 value1, XBit8 value2)},
	\item \emph{public static XBit16 or16bit(XBit16 value1, XBit16 value2)},
	\item \emph{public static XBit8 xor8bit(XBit8 value1, XBit8 value2)},
	\item \emph{public static XBit16 xor16bit(XBit16 value1, XBit16 value2)},
	\item \emph{public static XBit8 shift8bit(XBit8 value1, int position)},
	\item \emph{public static XBit8 shift16bit(XBit8 value1, int position)}.
\end{itemize}
są analogicznie do metody and8bit(XBit8 value1, XBit8 value2). Operacje bitowe wykonywane są za pomocą mechanizmów wbudowanych w język Java, a nadmiar bitów zostaje obcięty.
\chapter{Projekt aplikacji}
	uml-e, mvc. wzorce projektowe, może maven????
	
	\section{Podział aplikacji}
	Aplikację podzielono na 3 mniejsze moduły, Xbit, z80emu-core, z80emu-gui z czego każdy z nich jest osobnym mniejszym projektem. 
	
	\subsection{XBit}
	Java jest językiem programowania wysokiego poziomu, kompilowanym do kodu bajtowego. Z tego powodu nie jest on zazwyczaj stosowany w emulacji, gdyż kod emulatora musi być uruchamiany w maszynie wirtualnej, co nie należy do optymalnych rozwiązań.
	
	Innym poważnym problemem Javy jest brak typów danych przechowujących tylko wartości dodatnie. James Gosling, jeden z twórców Javy tak argumentuje ich brak: 
	,,Dla mnie jako projektant języków programowania, do których tak naprawdę ostatnimi czasy się nie zaliczam, coś prostego oznaczało skończenie na założeniu że losowy deweloper będzie w stanie zapamiętać specyfikacje. 
	Ta definicja mówi, że na przykład Java i wiele innych języków nie są proste, i w rzeczywistości wiele języków kończy z funkcjonalnościami których do końca nikt nie rozumie. Spytaj jakiegoś programistę języka C o dodatnie typy liczbowe, i odkryjesz że prawie żaden programista C faktycznie nie rozumie co dzieje się z typami bez znaku, czym jest arytmetyka liczb całkowitych. Takie rzeczy sprawiły że C jest językiem skomplikowanym. Myślę że w tej kwestii Java jest prosta"\cite{javaGoslingInterview}.
	
	
	Problem ten rozwiązano tworząc własną bibliotekę o roboczej nazwie XBit. Przechowuje ona liczby 8 i 16 bitowe, które mogą być interpretowane zarówno jako wartości całkowite jak i liczby w kodzie uzupełnień do dwóch. Biblioteka potrafi zwrócić konkretne bity, stworzyć reprezentacje liczby z pojedynczych bitów i typów prymitywnych, obudowuje operacje arytmetyczne z uwzględnieniem przepełnienia i przeniesienia. Kod projektu z jej użyciem staje się czytelniejszy i łatwiejszy do ewentualnej refaktoryzacji. 
	
	\subsection{z80emu-core}
	
	\subsection{z80emu-gui}
	
	
	